\documentclass{article}
\usepackage{geometry}
\usepackage{tcolorbox}

\geometry{a4paper, margin=1in}

\title{The UOR Repository Structure}
\author{UOR Foundation}
\date{TBD}

\begin{document}

\maketitle

\begin{abstract}
The UOR Foundation has outlined the following repository structure and hierarchy as the Universal Template for UOR repositories.
The \textit{intention} is for the orderly development and publication of the \textbf{UOR Kernel}, the \textbf{UOR Operating System}, and the \textbf{UOR Applications} that make up its associated modules, utilities, applications, and platform of the \textbf{UOR Computer}.
The \textit{aim} is to maintain consistency, scalability, and clarity across all repositories while enabling creative collaboration for evolution and governance the UOR Computer.

Following are the core templates for the UOR Foundation \& UOR Computer.
The unique function or capability of any file or code will determine the ``type'' of repository.

\begin{tcolorbox}[title=UOR Foundation Core Templates]
    \begin{enumerate}
        \item{UOR Foundation Repository}
        \item{UOR Kernel Repository}
        \item{UOR Utilities Repository}
        \item{UOR Operating System Repository}
        \item{UOR Applications Repository}
        \item{UOR Platform Repository}
    \end{enumerate}
\end{tcolorbox}
\end{abstract}

\section{UOR Foundation Repository}
\textbf{Purpose:} The UOR Foundation repository serves as the authoritative source for core foundational documents that direct and guide the Foundation.
This repository maintains the stability and integrity allowing for the creative development of the UOR Foundation \& it's source code.

\begin{verbatim}
uor-foundation/
├── docs/
│   ├── README.md        # Overview of the Foundation's purpose
│   ├── policies/        # Stable, guiding policy documents
│   │   ├── governance.md # Governance structure
│   │   ├── funding.md   # Funding policies
│   │   └── contributions.md # Contribution guidelines
│   ├── instructions/    # Detailed operational instructions
│   ├── history/         # Historical records and archives
│   └── LICENSE          # Licensing and legal documents
├── templates/           # Templates for policy or instructional documents
├── meta/                # Metadata about the Foundation's operations
├── .github/
│   └── workflows/       # CI/CD workflows for maintaining documentation
├── .gitignore
└── secure/              # Secure, access-controlled storage for sensitive documents
\end{verbatim}

\section{UOR Kernel Repository}
\textbf{Purpose:} Maintains and develops the UOR kernel, providing core functionality and system calls for the operating system and dependent repositories.

\begin{verbatim}
uor-kernel/
├ src/
│   ├── kernel/          # Core kernel code
│   ├── syscalls/        # System call definitions
│   └── arch/            # Architecture-specific code
├── docs/
│   ├── README.md        # Overview of the kernel
│   ├── API.md           # Syscall and kernel API descriptions
│   ├── CONTRIBUTING.md  # Contribution guidelines
│   └── LICENSE          # Licensing information
├── tests/
│   ├── unit/            # Unit tests for syscalls
│   ├── integration/     # Kernel integration tests
│   └── benchmarks/      # Performance benchmarks
├── configs/
│   └── default.config   # Default kernel configuration
├── .github/
│   └── workflows/       # CI/CD workflows
├── templates/
│   └── module_template/ # Template for kernel modules
├── .gitignore
└── meta/
    └── design_docs/     # Kernel design documents
\end{verbatim}

\section{UOR OS Modules Repository}
\textbf{Purpose:} Contains kernel extensions, drivers, or subsystems that integrate directly with the kernel.

\begin{verbatim}
uor-os-modules/
├── src/
│   ├── drivers/         # Device drivers
│   ├── filesystems/     # File system modules
│   └── networking/      # Networking stack
├── docs/
│   ├── module_api.md    # API documentation for modules
│   ├── CONTRIBUTING.md  # Contribution guidelines
│   └── LICENSE
├── tests/
│   ├── unit/            # Unit tests for modules
│   ├── integration/     # Kernel-module integration tests
└── .github/
    └── workflows/       # CI pipeline for module testing
\end{verbatim}

\section{UOR OS Utilities Repository}
\textbf{Purpose:} Stores core OS utilities such as commands and libraries.

\begin{verbatim}
uor-os-utilities/
├── src/
│   ├── commands/        # Utility commands (e.g., fetch, pull)
│   └── libraries/       # Shared utility libraries
├── docs/
│   ├── command_usage.md # Command usage instructions
│   ├── CONTRIBUTING.md
│   └── LICENSE
├── tests/
└── .github/
    └── workflows/
\end{verbatim}

\section{UOR Application Repository}
\textbf{Purpose:} Develops standalone applications that run on the kernel.

\begin{verbatim}
uor-app-<name>/
├── src/
│   ├── gui/             # Graphical user interface code
│   ├── backend/         # Application logic
│   └── syscalls/        # Interfacing with the UOR kernel
├── docs/
│   ├── user_guide.md    # How to use the application
│   ├── CONTRIBUTING.md
│   └── LICENSE
├── tests/
│   └── functional/      # Tests for user interaction
└── .github/
    └── workflows/
\end{verbatim}

\section{UOR Platform Repository}
\textbf{Purpose:} Serves as the hub for integrating and deploying applications built for the UOR OS.

\begin{verbatim}
uor-platform/
├── src/
│   ├── sdk/             # Developer SDKs for app creation
│   ├── deployment/      # Tools for packaging and deploying apps
├── docs/
│   ├── developer_guide.md
│   ├── platform_api.md
│   └── LICENSE
├── tests/
└── .github/
    └── workflows/
\end{verbatim}
\section*{Summary}
The proposed repository structure is designed to ensure:
\begin{itemize}
    \item \textbf{Consistency:} A unified template for all repositories.
    \item \textbf{Scalability:} Clear guidelines for creating new repositories.
    \item \textbf{Governance:} A hierarchy where the UOR kernel repository acts as the parent, defining standards and providing resources for child repositories.
    \item \textbf{Flexibility:} Repositories can adapt templates based on their specific needs while maintaining alignment with organizational goals.
    \item \textbf{Authority:} The UOR Foundation repository maintains the stability and guiding principles for all organizational operations and records.
\end{itemize}

This structure provides a solid foundation for managing the development and evolution of the UOR kernel, its ecosystem, and associated applications.

\end{document}

\end{document}

